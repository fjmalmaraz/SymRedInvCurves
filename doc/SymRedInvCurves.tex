\documentclass{article} 
\usepackage{amsmath,amsfonts,amsthm,amssymb} %AMS package
\usepackage{mathrsfs} % For numerical sets.
\usepackage{hyperref}
\usepackage[perpage,symbol*]{footmisc}
\def\atan{\mathop{\mathrm{atan}}}
\def\Re{\mathop{\mathrm{Re}}}
\def\Im{\mathop{\mathrm{Im}}}
\oddsidemargin 0cm
\textwidth 16cm 
\title{SymRedInvCurves:\\ Symmetric Reducible Invariant Curves }
\author{\`A.~Jorba \footnote{Universitat de Barcelona, Departament de
    Matem\`atiques i Inform\`atica  }, F.J. Mu\~noz-Almaraz\footnote{Universidad CEU-Cardenal Herrera, depto. Ciencias
  F\'{\i}sicas, Matem\'aticas y de la Computaci\'on }, J.C.~Tatjer\textsuperscript{*}}
\begin{document} 

\maketitle




This is a software to find 1--parameter families of symmetric invariant curves of a forced 1-D
maps with Lyapunov exponent constant. These 1-D maps are also called
skew--products and we study the following parametric systems
\begin{equation} \label{eq:system}
\left \{ 
\begin{array}{@{}l}
x_{n+1}=f(a\,x_n)+ b\, g(\theta_n) 
\\
\noalign{\smallskip}
\theta_{n+1}=\theta_n +\omega
\end{array} 
\right .
\end{equation} 
where $f$ is an odd function and $g$ is periodic, being both of them
analytical functions. The system is
invariant  with respect to the change of variable  $(\theta,x)\mapsto
(\theta+\pi,-x)$ and the symmetric curve is invariant w.r.t this
change, i.e. $x(\theta+1/2)=-x(\theta)$. 

A full description of the algorithm used in this repository  can be found in the
preprint~\cite{Angel16}.  These algorithm use the Discrete Fourier
Transform (DFT) and  the library
FFTW~\cite{FFTW} has been employed and its installation is required. 

The functions $f$ and $g$ are defined in the file
\texttt{cont\_complex.c} with 	homonymous name. Moreover, first and
second derivatives of $f$ have to be defined with the function names
\texttt{df} and \texttt{ddf}. Other parameters have been defined in
the file \texttt{parameter.h}



 

Newton's method is the main ingredient of continuation techniques
which implies to solve  linear systems in an iterative way. After reduction of a
1-D map around an approximated solution a system with the following
framework must be solved 

\begin{equation}\label{sist_exp_liap_const}
\left \{
\begin{array}{l}
h_y(\theta+\omega)=\lambda h_y(\theta) +h_a \,
u_a(\theta)+h_b\,u_b(\theta), \\
2 \int^{1/2}_0 d(\theta) h(\theta) \, d\theta=L.
\end{array}
\right .
\end{equation} 
where the unknown are $h_y$ a symmetric function, $h_a$ and $h_b$, and
the data are the functions $u_a$, $u_b$, $d$ and the number
$L$. Following the procedure given by~\cite{Angel16}, it is possible to
uncouple the system, getting a linear system whose unknowns are $h_a$
and $h_b$ and their coefficients are calculated according to the
function in Section~\ref{sec:ReducedCoefficient}.

\section{Description of parameters} 

The file \texttt{parameter.h} defines the folowing parameters: 
\begin{description}
\item[\texttt{OMEGA}] This is the shift of the angular variable in the
  system. The default value is the conjugate of the golden ratio,
  $\Phi= .618033988749894848204586834365$. 
\item[\texttt{EPS\_FLOQUET}]  This is the maximum error during the
  reduction procedure.
\item[\texttt{EPS}] This is the maximum error solving the
  equation. Meanwhile, new iterations are calculated to reduce the
  error. 
\item[\texttt{LE}] Lyapunov exponent constant for all the solutions in
  the continuation. 
\item[\texttt{INITIAL\_A}] Initial value of the parameter $a$ to start the
  continuation. 
\item[\texttt{INITIAL\_B}] Initial value of the parameter $b$ to start the
  continuation. 
\item[\texttt{INITIAL\_CONTINUATION\_STEP}] Initial step of
  continuation, after that the continuation algorithm adapt the step to garantee
  a fast convergence to the solution.
\item[\texttt{BD\_FILE}] The name of the file where data are saved for
  each calculated solution. The first column is the parameter $a$, the
  second column the parameter $b$, the third column is the length of
  the invariant  curve, the number of Newton's method iteration to converge,
  continuation step and error. The default value is \texttt{bd.dat}
\item[\texttt{TABU\_FILE}] The name of the file where every table
  values on a uniform mesh on $[0,1/2]$ for the solution
  is stored, separated by an empty line. The default value is \texttt{tabu.dat}
\item[\texttt{INITIAL\_SOLUTION}]  It assigns  the three first coefficients of
  the Fourier series of the starting solution.  
\item[\texttt{MODOS\_MAX}] It is the maximum number of Fourier
  coefficients allowed to perform the continuation. If more coefficients
  are needed to get a solution with the last 10\% of coefficient less
  than EPS*EPS*EPS, the program is stopped. 
\end{description}

\paragraph{Caviat:} The value of \texttt{MODOS\_MAX} must be small
enough to avoid exceeding the computer memory. 

\section{Fourier Series (\texttt{fourier} structure)}
Here we describe the file\verb@symmetric_curves.c@. There is a header
file associated with the same name.  A
structure is defined which contains the information about the
function. This structure consists of the following attributes: 
\begin{description}
  \item[\texttt{mod}] This is the  number of terms in the Fourier series
    describing the function.
  \item[\texttt{type}] There
exist two types \verb@SYMMETRIC@ (1) and \verb@HALF_PERIODIC@ (0). Some
methods have a different behaviour depending on this variable. 
 \item[\texttt{coef}] This is a complex vector containing the values of the
   Fourier series.
\item[\texttt{tab}] This is a vector with the values of the function on a
  uniform mesh in $[0,1/2]$
\item[\texttt{tabom}] This is a vector with the values of the function on a
  uniform mesh in $[\omega,1/2+\omega]$. Please, notice that the
  inteval is considered in the torus modulus 1. 
\end{description}
There are several methods which are applied to this structure:
\begin{description}
\item[\texttt{dimensiona}] Init the
vectors and increase the dimension if they were previously defined.
\item[\texttt{tabulacion}] Given the Fourier coefficients the values of the function on a
  uniform mesh in $[0,1/2]$ are calculated. 
\item[\texttt{coeficiente}] Given the value of the function on a mesh, the
  Fourier coefficients are described. 
\item[\texttt{coef\_integral}] Procedure to calculate the values of
  the coefficients of Trapezoidal rule. 
\end{description}
 
\subsection{ Type \texttt{SYMMETRIC}}
We consider $x:\mathbb{R}/\mathbb{Z}\to \mathbb{R}$ such that $x(\theta+\frac 1 2)=-x(\theta)$ 
The function $\theta\mapsto e^{i\pi \theta} x(\theta/2)$ is 1-periodic. Indeed,
\begin{equation}
e^{i\pi(\theta+1)}x(\frac{\theta+1}{2})=-e ^{i\pi\theta} x(\frac{\theta}{2}+\frac 1 2) =
e^{i\pi\theta} x(\theta/2).
\end{equation}
If $\tilde{\theta}_k$ is a uniform mesh of $[0,1]$, with $\tilde{\theta}_k=k/n$ 
There exists $\{Y_j\}_{j=0}^{n-1}$ whose FFT interpolates the function on the points $\tilde{\theta}_k$, i.e 
$\{Y_j\}_{j=0}^{n-1}$ verifies 
\begin{equation}
e^{i\pi\tilde{\theta}_k} x(\tilde{\theta}_k/2)=
\sum^{n-1}_{j=0} Y_j e^{-2\pi j\tilde{\theta}_k i} =
\sum^{n-1}_{j=0} Y_j e^{-i2\pi j k /n}=\mbox{DFT}(\{Y_j\}_{j=0}^{n-1})
\end{equation}
\subsubsection{Calculation of the function on a mesh $[0,1/2]$}
An uniform mesh of $[0,1/2]$ is given by $\theta_k=\tilde{\theta}_k/2$. So,
$e^{2i\pi\theta_k} x(\theta_k)=\mbox{DFT}(\{Y_j\}_{j=0}^{n-1})$.

\paragraph{Description of calculation of \texttt{fourier.tab}:} Given the coefficients of the Fourier series $ \{Y_j\}_{j=0}^{n-1}$,
if $X_k$ is the output of DFT, i.e.~$\{ X_k
\}_{k=0}^{n-1}=DFT(\{Y_j\}_{j=0}^{n-1})$. The value of the function
on $\theta_k$ is given by $e^{i\pi k/n}X_k$. 
%\begin{equation}
%\{Y_j\}_{j=0}^{n-1}\stackrel{DFT}{\longrightarrow}\{ X_k \}_{k=0}^{n-1}\stackrel{e^{i\pi k/n}X_k}{\longrightarrow}
%\{ x(\theta_k) \}_{k=0}^{n-1}
%\end{equation}

The points $ \{ x(\theta_k) \}_{k=0}^{n-1}$ are given by 
\begin{equation}
x(\theta_k)= \sum_{j=0}^{n-1} Y_j e^{-i2\pi(2j+1) \theta_k}
\end{equation}

\paragraph{Calculation of coefficients \texttt{fourier.coef}:} Given
the values  of the function  $ \{X_j\}_{j=0}^{n-1}$ on a mesh
$[0,1/2]$, we calculate the inverse DFT of $e^{2i\pi \, k /n}X_k$, i.e
$\{Y_j\}_{j=0}^{n-1}=DFT^{-1}(e^{2i\pi \, k /n})$. 


\subsubsection{Calculation of tabom}
The function $\tilde{\theta}\mapsto x(\tilde{\theta}/2)$ is approximate to 
\begin{equation} 
 e^{i\pi \tilde{\theta}}x(\tilde{\theta}/2)
\sim
\sum^{n/2-1}_{j=0} Y_j e^{-i 2\pi j \tilde{\theta}} +
\sum^{n/2-1}_{j=0} Y_{n-1-j} e^{i 2\pi (j+1) \tilde{\theta}}
\end{equation} 
With the change $ \theta=\frac{\tilde{\theta}}{2}$
\begin{equation}
x(\theta)\sim\sum^{n/2-1}_{j=0} Y_j e^{-i2\pi(2j+1) \theta} +
\sum^{n/2-1}_{j=0}Y_{n-1-j} e^{i2\pi(2j+1) \theta}
\end{equation}
Obviously, if $x(\theta)$ is well approximated, then
$\overline{Y_j}=Y_{n-1-j}$.
So, 
$x(\theta_k+\omega)$ is approximate to
\begin{equation}\label{aproximacion_theta_omega}
x(\theta_k+\omega)\sim 
\sum^{n/2-1}_{j=0} Y_j e^{-i2\pi(2j+1) \omega} e^{-i\pi(2j+1)k/n}+
\sum^{n/2-1}_{j=0} Y_{n-1-j} e^{i2\pi(2j+1) \omega} e^{i\pi(2j+1)k/n}
\end{equation}
We define  $Z_j:=Y_j e^{-i2\pi(2j+1) \omega}$ for $j=0,...,n/2-1$ and $Z_j=\overline{Z_{n-1-j}}$ for $j=n/2,...,n-1$.
 So,
\begin{equation}
\sum^{n/2-1}_{j=0} Y_{n-1-j} e^{i2\pi(2j+1) \omega} e^{i\pi(2j+1)k/n}=
\sum^{n/2-1}_{j=0} \overline{Z_j} e^{i\pi(2j+1)k/n}=
\sum^{n-1}_{j=n/2} Z_j e^{i\pi (2n-2j-1)k/n}= 
\sum^{n-1}_{j=n/2} Z_j e^{-i\pi (2j+1)k/n}
\end{equation}
With the Equation~\eqref{aproximacion_theta_omega} we have, 
\begin{equation}
x(\theta_k+\omega)\sim  \sum^{n-1}_{j=0} Z_j e^{-i\pi (2j+1)k/n}
= e^{-i\pi k/n}\sum^{n-1}_{j=0} Z_j e^{-i\pi (2j+1)k/n}=
\mbox{DFT}(\{Z_j\}_{j=0}^{n-1})
\end{equation}

\paragraph{Description of calculation of \texttt{fourier.tabom}:} Given the coefficients $\{Y_j\}_{j=0}^{n-1}$ of the
Fourier series, we apply the formula to define $Z_j$ and we apply
discrete Fourier transform (DFT). The output is the values of the
function on the mesh $[\omega,1/2+\omega]$
%\begin{equation}
%\{Y_j\}_{j=0}^{n-1} \stackrel{}{\longrightarrow} \{Z_j\}_{j=0}^{n-1}
%\stackrel{\mbox{DFT}}{\longrightarrow} \{ x(\theta_k+\omega)\}_{k=0}^{n-1}
%\end{equation}


\subsection{Type \texttt{HALF SYMMETRIC}}
Use  the Discrete Fourier Transform for a real function.  

\section{Coefficient    of a reduced system}
\label{sec:ReducedCoefficient}


Given the function  $r$  
$$\label{reduced_systemcoef}
y(\theta+\omega)=\lambda y(\theta) + r(\theta) \\
$$
We calculate an approximation of the integral $2 \int^{1/2}_0
d(\theta) y(\theta) \, d\theta$ 

This function is implemented in the function
\verb@coefficient_parameter_system@ the first parameter are the series
corresponding to $d(\theta)$ and the second parameter the series of
the independent term $r(\theta)$, the third parameter $\lambda$ and
the fourth parameter is number of modes $n$.

\begin{verbatim}
double coefficient_parameter_system(fourier *tint,fourier *rf,double lambda,double dmod)
\end{verbatim}

We consider the following series for every function
\begin{align}
y(\theta)& = \sum^{n/2-1}_{j=0} Y_j e^{-i2\pi(2j+1) \theta} +
\sum^{n/2-1}_{j=0}\overline{Y_{j}} e^{i2\pi(2j+1) \theta} \\
r(\theta)& = \sum^{n/2-1}_{j=0} R_j e^{-i2\pi(2j+1) \theta} +
\sum^{n/2-1}_{j=0}\overline{R_j} e^{i2\pi(2j+1) \theta} 
\end{align}
We substitute in~\eqref{reduced_systemcoef} and we get 
\begin{equation}
Y_j e^{-i2\pi(2j+1) \omega}= \lambda Y_j+R_j, \qquad \forall j=0,\ldots,\frac n 2 -1.
\end{equation}
So, the value of $Y_j$ is 
\begin{equation} \label{value_Yjcoef}
Y_j=\frac{R_j}{e^{-i2\pi(2j+1)\omega}-\lambda}
\end{equation} 
We estimate the second equation in~\eqref{reduced_systemcoef} with  Composite Trapezoidal Rule
\begin{equation}
\frac 1 n \sum^{n-1}_{k=0} 
d(\theta_k) y(\theta_k) = \gamma
\end{equation}
The function $y$ evaluated on $\theta_k$ is 
$y(\theta_k)=e^{-i\pi k/n}\sum_{j=0}^{n-1} Y_j e^{-i2\pi(2j+1)k/n}$.So, 
\begin{equation}
\frac 1 n 
\sum_{j=0}^{n-1} \left ( 
\sum_{k=0}^{n-1}  e^{-i\pi k/n} d(\theta_k)  e^{-i2\pi j k /n} 
\right ) Y_j=\gamma 
\end{equation}
We define 
\begin{equation}
L_j:= \sum_{k=0}^{n-1}  e^{-i\pi k/n} d(\theta_k) e^{-i2\pi j k /n} 
\end{equation}
Obviously, $\mbox{DFT}(\{ e^{-i\pi k/n} d(\theta_k) \}_{k=0}^{n-1} )=\{L_j\}_{j=0}^{n-1}$. 
We check that $L_{n-1-j}=\overline{L_j}$
\begin{equation}
L_{n-1-j}=\sum_{k=0}^{n-1} e^{-i\pi k /n} d(\theta_k) e^{-i2\pi(n-1-j)k/n}=
\sum_{k=0}^{n-1}d(\theta_k) e^{-i\pi k /n} e^{i2\pi k /n} e^{i2\pi jk/n}=\overline{L_j}
\end{equation}
With the above notation 
$$
2 \int^{1/2}_0 d(\theta) y(\theta) \, d\theta=\frac 1 n \sum_{j=0}^{n-1} L_jY_j = \frac 2 n \Re(\sum_{j=0}^{n/2-1} L_jY_j)= \frac 2 n \Re \left ( \displaystyle\sum_{j=0}^{n/2-1} \frac{L_j R_j}{e^{-i2\pi (2j+1)\omega}-\lambda}\right )
$$


\begin{thebibliography}{99}
\bibitem[1]{Angel16} A. Jorba, F.J. Mu\~noz, J.C. Tatjer \emph{On
    non-smooth pitchfork bifurcations in invertible
    queasi-periodically forced 1-D maps}. Preprint 
\bibitem[2]{FFTW} M.~Friggo, S.G.~Johnson  \emph{FFTW}  Available in \url{http://www.fftw.org/}
\end{thebibliography}
\end{document}


